\documentclass{beamer}

\usepackage{amsmath, amssymb, hyperref, graphics, wasysym, tikz}
%\usetikzlibrary{cd}
\usepackage{mathpazo}

\AtBeginDocument{
   \DeclareSymbolFont{AMSb}{U}{msb}{m}{n}
   \DeclareSymbolFontAlphabet{\mathbb}{AMSb}}

\newcommand{\AAA}{\mathbb{A}}
\newcommand{\C}{\mathbb{C}}
\newcommand{\Z}{\mathbb{Z}}
\newcommand{\R}{\mathbb{R}}
\newcommand{\Q}{\mathbb{Q}}

\title{MAS439 Lecture 11 \\ Algebraic Subsets}

\date{November 3rd}

\begin{document}

\begin{frame}
\titlepage
\end{frame}

\begin{frame}{Starting Geometry}
We've been studying polynomial rings algebraically, from the viewpoint of their universal property.  We now begin to study them from another angle, as rings of functions on affine space.

\begin{definition} Let $k$ be a field.  Define affine space
$$\AAA_k^n=\{(a_1,\dots, a_n)\in k^n\}$$
\end{definition} 

Note that as a set, this is just $k^n$.  The notation highlights the fact that we are not going to view $\AAA_k^n$ as a ring or a vector space, but as a geometric space that has functions on it.

\end{frame}


\begin{frame}{Vanishing loci}

\begin{definition} Let $I\subset k[x_1,\dots, x_n]$ be an ideal.  Define:

$$V(I)=\{p\in \AAA^n_k:f(p)=0\forall f\in I\}$$
\end{definition}

Note that $V(X)$ makes sense for any subset $X\subset k[x_1,\dots, x_n]$.  But note that $V(X)$ agrees with $V(I)$, where $I=(X)$ is the ideal generated by elements of $X$.  So we can work with only ideals instead.

\\~\\

In the other direction, since $k[x_1,\dots, x_n]$ is Noetherian, any ideal $I$ is generated by finitely many elements $f_1,\dots, f_k$, and so $V(I)$ is the locus where all the $f_i$ vanish simultaneously.

\end{frame}


\begin{frame}{Ideals of functions that vanish}
We now look at the other direction:

\begin{definition}
Given any subset $X\subset \AAA_k^n$, define

$$I(X)=\{f\in k[x_1,\dots, x_n]:f(p)=0\quad\forall p\in V\}$$

\end{definition}

Note that as the name suggests, $I(V)$ is an ideal.  Why?


\end{frame}


\begin{frame}{$V$ and $I$ are order reversing}

Suppose that $I\subset J$, then $V(J)\subset V(I)$.  (Why?)
\\~\\
Similarly, suppose that $X\subset Y\subset \AAA_k^n$.  Then $I(Y)\subset I(X)$.


\end{frame}


\begin{frame}{Algebraic subsets}

\begin{definition}
A subset $X\subset \AAA_k^n$ is \emph{algebraic} if it is of the form $V(I)$ for some ideal $I\subset k[x_1,\dots, x_n]$
\end{definition}

Now we'll look at a lot of examples (see the notes)


\end{frame}

\begin{frame}{Unions and intersections}
\alert{Tom's notes state this backwards!}

Finite unions and arbitrary intersections of algebraic subsets are algebraic, and we can describe their ideals:

\begin{lemma}
Given an arbitrary collection of ideals $I_s$ indexed by a set $S$, we have:

$$V\left(\sum_{s\in S} I_s\right)=\bigcap_{s\in S} V(I_s)$$

Given two ideals $I, J$, we have 
$$V(I\cap J)=V(I)\cup V(J)$$
\end{lemma}

\end{frame}

\begin{frame}{Examples}
Let $I=(x,y)\in\C[x,y,z]$, and $J=(z)$.  
\\~\\
Let's check the lemma in this case:
What's $V(I)$, $V(J)$, $I+J, I\cap J$


\end{frame}

\begin{frame}{Reminder on point-set topology}

\begin{definition}
Let $X$ be a space.  A \emph{topology} $\mathcal{T}$ on $X$ is a collection of subsets of $X$ (called \emph{open} sets such that:
\begin{enumerate}
\item $X$ and $\emptyset$ are open
\item Finite intersections of open sets are open:
$$U,V\in\mathcal{T}\implies U\cap V\in\mathcal{T}$$
\item Arbitrary unions of open sets are open:
$$U_s\in\mathcal{T}\forall s\in S\implies \bigcup_{s\in S} U_s\in\mathcal{T}$$
\end{enumerate}


\end{definition}

\end{frame}

\begin{frame}{Motivating example}
Let $(X,d)$ be a metric space.  (e.g., $\R^n$ with its usual distance function).
Define a subset $U$ to be \emph{open} if for all $x\in X$, there exists an $r>0$ such that
$$B_r(x)=\{y\in X : d(x,y)<r\}\subset U$$



\end{frame}

\begin{frame}{Weirder examples of topologies}
Let $X$ be any set.  

\begin{block}{Discrete topology}
Define any subset of $X$ to be open.
\end{block}
\begin{block}{Concrete topology}
Define only $X$ and $\emptyset$ to be open.
\end{block}

\begin{block}{One more}
Define $U\subset X$ to be open if and only the complement of $U$ is finite
\end{block}

\end{frame}

\begin{frame}{Algebraic subsets are closed sets}
\begin{definition}
If $X$ is a topological space, we call a subset $U\subset X$ \emph{closed} if its complement $U^c$ is open.
\end{definition}
\alert{Warning:} A set can be both open and closed (clopen).  (There's a \emph{downfall} parody video about this.
\\~\\
Since open sets are closed under finite intersection and arbitrary union, closed sets are closed under arbitrary union and finite intersection.
\\~\\
Thus, our lemma shows that the algebraic subsets form the closed sets of a topology on $\AAA_k^n$, called the \emph{Zariski} topology.


\end{frame}


\end{document}
