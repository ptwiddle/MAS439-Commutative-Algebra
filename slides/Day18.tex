\documentclass{beamer}

\usepackage{amsmath, amssymb, hyperref, graphics, wasysym, tikz}
%\usetikzlibrary{cd}
\usepackage{mathpazo}

\AtBeginDocument{
   \DeclareSymbolFont{AMSb}{U}{msb}{m}{n}
   \DeclareSymbolFontAlphabet{\mathbb}{AMSb}}

\newcommand{\AAA}{\mathbb{A}}
\newcommand{\C}{\mathbb{C}}
\newcommand{\Z}{\mathbb{Z}}
\newcommand{\R}{\mathbb{R}}
\newcommand{\Q}{\mathbb{Q}}

\title{MAS439 Lecture 18 \\ Categories}

\date{December 7th}

\begin{document}

\begin{frame}
\titlepage
\end{frame}

\begin{frame}{Why categories?}

\begin{block}{Viewpoint:}

We've been implicitly thinking ``categorically'' by using Universal properties, and thinking about $\C[x]$ and $\C[x,y]/I$ in terms of maps out of them.\\~\\
\end{block}

\begin{block}{Language:}

We've established a dictionary between geometry and algebra, taking algebraic subsets to their coordinate ring, and maps between spaces to maps between rings. The dictionary can seem a little unwieldy and abstract at first, but it's actually a common type occurence, and category theory is language invented to discuss just such things,
\end{block}


\end{frame}

\begin{frame}{Definition of a category}

A category $\mathcal{C}$ consists of:

1. A collection of \emph{objects} $\text{Ob}(\mathcal{C})$ (think: objects are sets}
2. For each pair of objects $A,B\in\text{Ob}(\mathcal{C})$, a set of \emph{morphisms} $\Hom_{\mathcal{C}}(A, B)$ (think: morphisms are maps between sets)
3. For every triple of objects $A, B,C$ a composition map

\end{frame}


\end{document}
