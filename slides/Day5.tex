\documentclass{beamer}

\usepackage{amsmath, amssymb, hyperref, graphics, wasysym}
\usepackage{mathpazo}

\newcommand{\C}{\mathbb{C}}
\newcommand{\Z}{\mathbb{Z}}
\newcommand{\R}{\mathbb{R}}

\title{MAS439 Lecture 5 \\ Quotient Rings}

\date{October 12th}

\begin{document}

\begin{frame}
\titlepage
\end{frame}


\begin{frame}{}

Recall an ideal $I\subset R$ was a subset that was closed under addition, and closed under multiplication by elements of $R$; in shorthand:
$$I+I\subset I$$
$$R\cdot I\subset I$$

Today, given an ideal $I\subset R$, we will define a \emph{quotient ring} $R/I$.

Tomorrow, we will prove the \emph{first isomorphism theorem}, which at ti's simplest level says that given any homomorphism $\varphi:R\to S$, we have $\textrm{Im}(\varphi)\cong R/I$.

\end{frame}

\begin{frame}{A first example: $\mathbb{Z}/n\mathbb{Z}$}
We've seen that the ideals of $\mathbb{Z}$ are precisely the principal ideals $(n)=n\mathbb{Z}$.

Thus $\mathbb{Z}/(n)=\mathbb{Z}/\mathbb{Z}$.

\begin{block}{Something to keep in mind:}
We often \emph{think} ``$\mathbb{Z}/n\mathbb{Z}=\{0,1,2,\dots, n-1\}$.''

This isn't quite right, really:

$$\mathbb{Z}/n\mathbb{Z}=\{k+n\mathbb{Z} \}$$

We do this because awkward to think of ring elements as being themselves sets; and things in the second description have more than one name, i.e., $2+7\mathbb{Z}=-5+7\mathbb{Z}$.

\end{block}


\end{frame}


\begin{frame}{You've seen this before: quotient groups}
Recall that, given a normal subgroup $N \subset  G$, we have the quotient subgroup $G/N$.  The elements of $G/N$ are the \emph{cosets} of $N$ -- sets of the form $gN$.  Alternatively, elements of $G/N$ are equivalence classes, where $g\sim h$ if $gh^{-1}\in N$.

\begin{block}{Why did $N$ need to be normal?}
To make multiplication well defined.

\end{block}

\end{frame}





\begin{frame}{Definition of $R/I$ as a set}

As a set, the quotient ring $R/I$ is defined to be the set of equivalence classes under the relation $r\sim s$ if $r-s\in I$.

If $r\in R$ any element, and $i\in I$ any element, we see that $r+i\sim r$.  Furthermore, if $s\sim r$, then $s-r=i\in I$, and so $s=r+i$.  Thus, we see that the equivalence classes of $\sim$ are exactly the cosets of $I$ -- sets of the form $r+I$.


\end{frame}


\begin{frame}{Operations on $R/I$}
We have defined what $R/I$ is as a set; we now need to turn $R/I$ into a ring.  We define addition and multiplication on $R/I$ by adding/multiplying representatives from the equivalence classes.  That is, 

$$[a]+[b]=[a+b]$$
$$[a]\cdot [b]=[a\cdot b]$$

\begin{block}{To do list:}
\begin{itemize}
\item Check that these operations are well defined
\item Check that these operations satisfy the axioms of a ring
\end{itemize}
\end{block}

\end{frame}

\begin{frame}{Addition is well defined}
Suppose we chose $a^\prime\sim a$ and $b^\prime \sim b$.  For addition to be well defined we need:
$$[a^\prime+b^\prime]:=[a^\prime]+[b^\prime]=[a]+[b]=:[a+b]$$

\begin{itemize}
\item Since $a^\prime \sim a$, we have $a^\prime-a=i\in I$
\item Since $b^\prime \sim b$, we have $b^\prime-b=i\in I$ 
\item $(a^\prime+b^\prime)-(a+b)=(a^\prime-a)+(b^\prime-b)=i+j$
\item Since $I$ closed under addition, $i+j\in I$, so $(a^\prime+b^\prime)\sim (a+b)$
\end{itemize}

\end{frame}

\begin{frame}{Multiplication is well defined}
Suppose $$a^\prime-a=i\in I,\qquad b^\prime-b=j\in I$$  

We need to show that $$a^\prime\cdot b^\prime-a\cdot b\in I$$

Then:


$$a^\prime\cdot b^\prime-a\cdot b=(a+i)\cdot (b+j)-a\cdot b=a\cdot j+b\cdot i+i\cdot j$$

\begin{itemize}
\item Since $i, j\in I$ and $I$ an ideal, we have $a\cdot i, b\cdot j, i\cdot j\in I$. 
\item  Since $I$ is an ideal, their sum is also in $I$. 
\item  Hence $a^\prime\cdot b^\prime\sim a\cdot b$ and multiplication is well defined.
\end{itemize}

\end{frame}

\begin{frame}{$R/I$ satisfies the ring axioms}

These proofs are all just symbol pushing.  For instance, to show that the distributive law holds, we have:

\begin{align*}
([a]+[b])\cdot [c]& =[a+b]\cdot[c] \\
& =[(a+b)\cdot c] \\
&=[a\cdot c+b\cdot c] \\
&=[a\cdot c]+[b\cdot c]=[a]\cdot [c]+[b]\cdot [c]
\end{align*}
\end{frame}


\begin{frame}{In words}

To me, that last proof was rather unenlightening.

The ring axioms are satisifed in $R/I$ because the operations $+, \cdot$ are defined in terms of lifting to representatives in $R$; and the axioms hold there.



\end{frame}


\begin{frame}[plain,c]

\begin{center}

\Huge

\usebeamercolor[fg]{frametitle}
$\twonotes$ Let's all go to the lobby $\twonotes$ \\ $\twonotes$ Let's all go to the lobby $\twonotes$ \\
(2 minute intermission)
\end{center}

\end{frame}

\begin{frame}{Example: $\R[x]/(x^2)$}

First, we have to understand it as a set -- we want to give a \emph{unique} name to each element of $R/I$.  This is usually done by picking a representative from each coset in some systematic way.

$I$ consists of linear combinations of monomials of degree 2 or bigger.  So every equivalence class contains exactly one linear term $a+bx$.  We see that 

$$[a+bx]\cdot [c+dx]=[ac+adx+bcx+adx^2]=[ac+(ad+bc)x]$$


\end{frame}


\begin{frame}{Example: $\C\cong\R[x]/(x^2+1)$}

\begin{block}{The division algorithm gives unique representatives}

Any polynomial $p(x)$ can be written uniquely as

 $$p(x)=(x^2+1)q(x)+bx+a$$. 

This means that $[p(x)]=[bx+a]$, so every class can be represented by a linear polynomial; furthermore, this representation is unique.
\end{block}

It's clear $[a+bx]+[c+dx]=[a+c+(b+d)x]$.
\end{frame}

\begin{frame}{Example: $\mathbb{C}=\R[x]/(x^2+1)$}

\begin{block}{Multiplication of representatives}
$$[a+bx]\cdot [c+dx]=[ac+(ad+bc)x+bdx^2]$$

But this isn't linear;  we need to get rid of the $x^2$ term.  Note that $bdx^2=bd(x^2+1)-bd$, and so $[bdx^2]=[-bd]$.

Thus, we see
$$[a+bx]\cdot [c+dx]=[ac-bd+(ad+bc)x]$$
which, if we replace $x$ with $i$, is exactly the formula for multiplying complex numbers.
\end{block}


\end{frame}

\begin{frame}{Constructing $\mathbb{F}_4$}

We claim that $R=\mathbb{F}_2[x]/(x^2+x+1)$ is a field with 4 elements.  

Exactly as in the last two examples, the division algorithm gives every equivalence class has a unique linear representative $a+bx$; now $a,b\in\mathbb{F}_2$, so there are indeed four elements.

We check:
$$[x]\cdot [x+1]=[x^2+x]=[1]$$
So every nonzero element has an inverse, and so $R$ is a field.
\end{frame}




\end{document}
