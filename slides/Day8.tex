\documentclass{beamer}

\usepackage{amsmath, amssymb, hyperref, graphics, wasysym, tikz}
\usetikzlibrary{cd}
\usepackage{mathpazo}

\newcommand{\C}{\mathbb{C}}
\newcommand{\Z}{\mathbb{Z}}
\newcommand{\R}{\mathbb{R}}
\newcommand{\Q}{\mathbb{Q}}

\title{MAS439 Lecture 8 \\ $k$-algebras}

\date{October 20th}

\begin{document}

\begin{frame}
\titlepage
\end{frame}





\begin{frame}{Forgotten bit from last time}

\begin{lemma} Let $\mathfrak{p}\subset S$ be a prime ideal, $\varphi:R\to S$ a homomorphism.  Then $\varphi^{-1}(\mathfrak{p})$ is prime
\end{lemma}


\\~\\

Similarly, the pull-back of a radical ideal is radical.  However:

\begin{example} Consider the inclusion $i:\mathbb{Z}\hookrightarrow\Q$.  Since $\Q$ is a field, $\{0\}$ is a maximal ideal in $\Q$.  However, $i^{-1}(\{0\})=\{0\}$ is not a maximal ideal of $\Z$.  
\end{example}
\\~\\
Hence, the pullback of a maximal ideal need not be maximal.
\end{frame}


\begin{frame}[plain, c]

\huge
\begin{center}


{\usebeamercolor[fg]{frametitle} Today's goal: \\~\\}

Understand the statement ``$\C[x,y]$ is a finitely generated $\C$-algebra''

\end{center}
\end{frame}



\begin{frame}{Why algebras?}

\begin{block}{$\C[x,y]$ is \alert{NOT} a finitely generated ring!}


But this is ``just'' because $\C$ is not finitely generated.  We have made our peace with $\C$, and are no longer scared of it ($\R$, really).  If we are willing to take $\C$ for granted, then to get $\C[x,y]$ we just need to add $x$ and $y$. A primary purpose of introducing $\C$-algebras is to make this idea precise.  
\end{block}\\~\\
\begin{block}{Never leave home without an algebraically closed field} 
We want to build in an (algebraically closed) field into our rings. $\C$-algebras do just that.
\end{block}

 
\end{frame}

\begin{frame}[fragile]{Formal definition of $k$-algebra}

Let $k$ be any commutative ring. 

\begin{definition} A $k$-algebra is a pair $(R,\phi)$, where $R$ is a ring and $\phi:k\to S$ a morphism.
\end{definition}


\begin{definition} A \emph{map of $k$-algebras} between $f:(R,\phi_1)\to (S,\phi_2)$ is a map of rings $f:R\to S$ such that $\phi_2=f\circ\phi_1$, that is, the following diagram commutes:
\begin{center}
  \begin{tikzcd}[column sep=small]
 & k  \arrow[rd, "\phi_2"] & \\
    R \arrow[rr, "f"] \arrow[<-,ru,  "\phi_1" ]&  & S
\end{tikzcd}
\end{center}



\end{definition}








\end{frame}


\begin{frame}[fragile]{Important examples of $k$-algebras}

\begin{itemize}
\item $k[x]$ is a $k$-algebra, with $\phi:k\to k[x]$ the inclusion of $k$ as constant polynomials.
\item $\C$ is an $\R$-algebra, with $\phi:\R\to \C$ the inclusion
\item $\C$ is also a $\C$-algebra, with $\phi:\C\to \C$ the identity

\item The ring $\textrm{Fun}(X,R)$ of functions is an $R$-algebra, with $\phi:R\to\textrm{Fun}(X,R)$ the inclusion of $R$ as the set of constant functions 

\item As there is a unique homomorphism $\phi:\Z\to R$ to any ring $R$, we see that any ring $R$ is a $\Z$-algebra in a unique way -- that is, rings are the same thing as $\Z$ algebras.
\end{itemize}
\end{frame}

\begin{frame}{Examples of maps of $k$-algebras}
\begin{itemize}



\item Complex conjugation from $\C$ to itself is a map of $\R$-algebras but NOT a map of $\C$-algebras.
\item If $R$ is a $k$-algebra, and $I$ an ideal, $R/I$ is a $k$-algebra, and the quotient map $R\to R/I$ is a morphism of $k$-algebras
\item A $\Z$-algebra map is just a ring homomorphism



\end{itemize}



\end{frame}


\begin{frame}{Slogan: Algebras are rings that are vector spaces}

We will usually take $k$ to be a field.  This has the following consequences:

\begin{itemize}
\item As maps from fields are injective, we have that $\phi:k\to R$ is injective, and so $k\subset R$ is a subring.
\item The ring $R$ becomes a vector space over $k$, with structure map $\lambda\cdot_{vs} r=\phi(\lambda)\cdot_R r$
\item Multiplication is linear in each variable: if we fix $s$, then $r\mapsto r\cdot s$ and $r\mapsto s\cdot r$ are both linear maps.
\item Going backwards, if $V$ is a vector space over $k$, with a bilinear, associative multiplication law and a unit $1_V$, then $V$ is naturally a $k$-algebra, with structure map $\phi:k\to V$ defined by $\lambda\mapsto \lambda\cdot 1_V$

\end{itemize}

\end{frame}

\begin{frame}{Finite-dimensional algebras }
\begin{definition} Let $k$ be a field.  We say a $k$-algebra $R$ is \emph{finite dimensional} if $R$ is finite dimensional as a $k$-vector space.
\end{definition}

\begin{example}
\begin{itemize}
\item $\C$ is a two dimensional $\R$-algebra
\item $\C[x]/(x^n)$ is an $n$-dimensional $\C$-algebra
\item $\C[x]$ is \alert{not} a finite-dimensional $\C$ algebra
\end{itemize}
\end{example}
\end{frame}

\begin{frame}{Toward finitely generated $k$-algebras}
Being finite dimensional is too strong a condition to place on $k$-algebras for our purposes.  We now define what it means to be finitely generated.  \\~\\

This is completely parallel to how we defined finitely generated for rings.


\end{frame}

\begin{frame}{Subalgebras}
\begin{definition} Let $(R,\phi)$ be a $k$-algebra.  A $k$-subalgebra is a subring $S$ that contains $\textrm{Im}(\phi)$.  
\end{definition}

\begin{itemize}
\item if $S$ is a $k$-subalgebra, then in particular it is a $k$-algebra, where we can use the same structure map $\phi$
\item The inclusion map $S\hookrightarrow R$ is a $k$-algebra map
\item To check if a subset $S\subset R$ is a subalgebra, we must check it is closed under addition and multiplication, and containts $\phi(k)$.
\end{itemize}


\end{frame}


\begin{frame}{Generating subalgebras}

\begin{definition} Let $R$ be a $k$-algebra, and $T\subset R$ a set.  The subalgebra generated by $T$, denoted $k[T]$, is the smallest $k$-subalgebra of $R$ containing $T$
\end{definition}

\begin{lemma} The elements of $k[T]$ are precisely the $k$-linear combinations of monomials in $T$; that is, elements of the form
$$\sum_{i=1}^m \lambda_i m_i$$
where $\lambda_i\in \phi(k)$ and $m_i$ is a product of elements in $T$
\end{lemma}
\end{frame}

\begin{frame}{Three ways of generating}

Let $T\subset \C[x]$ be the single element $x$.  Then
\begin{itemize}
\item The subring generated by $x$, written $\langle x\rangle$ is polynomials with integer coefficients: $\langle x\rangle=\Z[x]\subset \C[x]$.
\item The ideal generated by $x$, written $(x)$, are all polynomials with zero constant term
\item The $\C$-subalgebra generated by $T$ is the full ring $R=\C[x]$.
\end{itemize}
\end{frame}

\begin{frame}{$\C[x,y]$ is a finitely generated $\C$-algebra}
\begin{definition}
We say that a $k$-algebra $R$ is finitely generated if we have $R=k[T]$ for some finite subset $T\subset R$.
\end{definition}
\\~\\
Indeed, we have $\C[x,y]$ is generated as a $\C$ algebra by $\{x,y\}$.


\end{frame}
\end{document}
