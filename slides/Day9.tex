\documentclass{beamer}

\usepackage{amsmath, amssymb, hyperref, graphics, wasysym, tikz}
\usetikzlibrary{cd}
\usepackage{mathpazo}

\newcommand{\C}{\mathbb{C}}
\newcommand{\Z}{\mathbb{Z}}
\newcommand{\R}{\mathbb{R}}
\newcommand{\Q}{\mathbb{Q}}

\title{MAS439 Lecture 9 \\ Polynomial rings}

\date{October 26th}

\begin{document}

\begin{frame}
\titlepage
\end{frame}

\begin{frame}{Universal Property of $k[x]$}

\begin{lemma} Let $(R,\phi)$ be any $k$-algebra, and let $r\in R$ any element.  Then there exists a unique homomorphism of $k$-algebras $f:k[x]\to R$ such that $f(x)=r$.
\end{lemma}


\begin{proof}
\onslide<2->{
\begin{itemize}
\item<3-> Since $f$ is a $k$-algebra homomorphism, $f$ must send $k$ to $k$; i.e., if $a\in k\subset k[x]$, then we must have $f(a)=\phi(a)$.
\item<4-> By assumption, we know $f(x)=r$.
\item<5-> Since $f$ preserves addition and multiplication, we have:
$$f(a_nx^n+a_{n_1}x^{n-2}+\cdots+a_0)=$$
$$\phi(a_n)r^n+\phi(a_{n-1})r^{n-2}+\cdots +\phi(a_0)
$$
\end{itemize}
\onslide<6->{So $f$ just substitutes $r$ for $x$, and there can be at most one such homomorphism; it's easy to check that $f$ actually is a homomorphism.
}}
\end{proof}

\end{frame}








\begin{frame}{Universal objects characterize properties}

If $X$ and $Y$ have the same universal property, then typically it will follow, simply from the definition of the Universal Property, that $X$ and $Y$ are isomorphic.

\begin{lemma} Suppose $(X,x)$ and $(Y,y)$ both satisfy the Universal Property of the polynomial algebra over $k$.  Then there is a unique isomorphism $\varphi:X\to Y$ with $\varphi(x)=y$.
\end{lemma}

\end{frame}


\begin{frame}{Proof that Universal Property characterizes $k[x]$}

\begin{block}{How do we get a map from $X$ to $Y$?}

\onslide<2->{ \alert{The Universal Property:}} \onslide<3->{Since $X$ satisfies the universal property, there is a unique $k$-algebra homomorphism $\varphi$ with $\varphi(x)=y$. }
\end{block}

\onslide<4->{\begin{block}{How do we show $\varphi$ is an isomorphism?}
\onslide<5->{Construct an inverse $\psi:Y\to X$.}
\end{block}}

\onslide<6->{\begin{block}{How do we construct $\psi$?}
\onslide<7->{\alert{The Universal Property:}} \onslide<8->{ If $\psi$ is inverse to $\varphi$ we must have $\psi(y)=x$.  Since $Y$ satisfies the UP, there is a unique $\psi:Y\to X$ with $\psi(y)=x$.}  
\end{block}}
\end{frame}




\begin{frame}{Universal property characterizes $k[x]$, continued}

\begin{block}{How do we prove that $\psi\circ\varphi=\textrm{Id}_X$?}
\onslide<2->{\alert{The Universal Property}}\onslide<3->{ Applying the Universal Property of $(X,x)$ to itself, we see there is a \alert{unique} morphism $f:X\to X$ with $f(x)=x$. 

Clearly, the identity map $\textrm{Id}_X$ is such a map.

On the other hand, $\psi\circ\varphi:X\to X$, and  $\psi\circ\varphi(x)=\psi(y)=x$. 

Since $f$ was unique, we must have $\psi\circ\varphi=\textrm{Id}_X$.}
\end{block}

\onslide<4->{\begin{block}{How do we prove that $\varphi\circ\psi=\textrm{Id}_Y$}
Exactly the same, i.e.: \alert{The Universal Property} for $Y$.
\end{block}}

\end{frame}





\begin{frame}{Polynomial rings in more than one variable}
Tom spends most of a page defining $k[x_1,\dots, x_n]$.  Are we comfortable with what this is?

$k$ linear combinations of 

Monomials: $x_1^{e_1}x_2^{e_2}\cdots x_n^{e_n}$



\end{frame}



\begin{frame}

\begin{lemma} For any $n$, there is an isomorphism of $k$-algebras:
$$k[x_1,\dots,x_{n-1}][x_n]\cong k[x_1,\dots, x_n]$$
\end{lemma}
\begin{proof}
To go from the left to the right we multiply out; to go from the right to the left we factor out $x_n^k$'s.
\end{proof}

\end{frame}





\begin{frame}{Universal property for $k[x_1,\dots, x_n]$}

\begin{lemma} Let $(R,\phi)$ be a $k$-algebra, and let $r_1,dots, r_n$ be (not necessarily distinct) elements of $R$.  Then there exists a unique $k$-algebra homomorphism $f:k[x_1,\dots, x_n]$ such that $f(x_i)=r_i$.
\end{lemma}

One proof is to just adapt the proof we gave in the one variable case.  Being a $k$-algebra homomorphism and knowing $f(x_i)=r_i$ forces $f$ to be the map that substitutes $r_i$ in for $x_i$, and this is indeed a $k$-algebra homomorphism. 
\\~\\ 

Another proof uses induction on $n$, and is perhaps instructive to work through.

\end{frame}

\begin{frame}{Proof the multivariables UP via induction}

\begin{block}{Base Case: n=1}
We already did.
\end{block}

\begin{block}{Inductive step}
Suppose we know the universal property for $k[x_1,\dots, x_{n-1}]$.  We now prove it for $k[x_1,\dots, x_{n-1},x_n]$.  The key idea is to use the isomorphism $k[x_1,\dots, x_n]\cong k[x_1,\dots, x_{n-1}][x_n]$.

\end{block}
\end{frame}

\begin{frame}

\begin{block}{Main idea:}
Use the Universal property of $k[x_1,\dots, x_{n-1}][x_n]$ among $k[x_1,\dots, x_{n-1}]$ algebras.  
\end{block}

\begin{block}{First problem:}
To use the UP to get a map to $R$, we need $R$ to be a $k[x_1,\dots, x_{n-1}]$-algebra.
\end{block}

\begin{block}{Solution:}
From the $n-1$ variables UP for $k$ algebras, we have a unique $k$-algebra homomorphism $\phi$ from $k[x_1,\dots, x_{n-1}]$ to $R$ sending $x_i$ to $r_i$.  

We now use this map $\phi$ as the structure map to view $R$ as a $k[x_1,\dots, x_{n-1}]$ algebra.  
\end{block}


\end{frame}


\begin{frame}



Then, using the 1-variable UP for $k[x_1,\dots, x_{n-1}][x_n]$, there exists a unique $k[[x_1,\dots, x_{n-1}]$-algebra homomorphism $f:k[x_1,\dots, x_n]\to R$ with $f(x_n)=r_n$.   \\~\\

The map $f$ is a $k$-algebra homomorphism and has $f(x_i)=r_i$.  

\begin{block}{Need to show $f$ unique with these properties}

But unraveling the definitions, being a $k$-algebra homomorphism sending $x_i$ to $r_i$ for $1\leq i\leq n-1$ is exactly the same as being a $k[x_1,\dots, x_{n-1}]$-algebra homomorphism.  So, the map $f$ is the unique $k$-algebra homomorphism because it is the unique $k[x_1,\dots, x_{n-1}]$-algebra homomorphism.
\end{block}
\end{frame}


\begin{frame}{An important corollary of the Universal Property}

\begin{lemma} Let $R$ be a finitely generated $k$-algebra.  Then $R$ is the quotient of a polynomial ring: $R\cong k[x_1,\dots, x_n]/I$ for some ideal $I$.
\end{lemma}

\begin{proof} 
\begin{itemize}[<+->]

\item Since $R$ is finitely generated, suppose it is generated by $r_1,\dots, r_n$.  

\item By the Universal Property of polynomial algebras, there is a $k$-algebra homomorphism $f:k[x_1,\dots, x_n]$ sending $x_i$ to $r_i$.  

\item Since $\textrm{Im}(f)$ is a subring containing $k$ and the $r_i$, we have $R=k[r_1,\dots, r_n]\subset \textrm{Im}(f)\subset R$, and so $\textrm{Im}(f)=R$.

\item Then, by the first isomorphism theorem, we have $R=k[x_1,\dots, x_n]/\ker(f)$.
\end{itemize}



\end{proof}


\end{frame}


\end{document}
