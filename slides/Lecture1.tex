\documentclass{beamer}

\usepackage{amsmath, amssymb, hyperref, graphics}
\usepackage{mathpazo, soul}

\newcommand{\C}{\mathbb{C}}
\newcommand{\Z}{\mathbb{Z}}


\title{Commutative Algebra MAS439 \\ Lecture 1}
\author{Paul Johnson \\ \href{mailto:paul.johnson@sheffield.ac.uk}{paul.johnson@sheffield.ac.uk} \\ Hicks J06b}
\date{October 3rd}

\begin{document}

\begin{frame}
\titlepage
\end{frame}


\begin{frame}{Assessment is entirely via problem sets}

\begin{itemize}
\item Five problem sets throughout term, due Friday at 10
\item Planned for Weeks 3, 5, 8, 10, and 12
\item You are encouraged, but not required, to write your solutions in \LaTeX
\item You are encouraged, but not required, to work together in groups of 2 or 3
\item Of course life happens; if there's an issue with handing an assignment in on time let my know as soon as possible
\end{itemize}

In previous years we had problem sets every week and it was a bit intense; on other hand they could drop lowest ones.

\end{frame}


\begin{frame}{Wait, groupwork?! How does that work?}

\begin{itemize}
\item Each group member writes up and hands in their own solution
\item If you do work in groups, please write who you worked with on every assignment
\end{itemize}

\begin{block}{What is/isn't allowed:}
\begin{itemize}
\item You should \alert{NOT} be writing up identical solutions, or even writing up your solutions sitting together.  
\item Rather, in the group digest what the problem is actually asking, come up with an informal / pseudo-formal solution
\item \alert{LATER}, on your own, write up the full, rigorous solution
\end{itemize}
\end{block}

\end{frame}

\begin{frame}{Rigour and intuition, proof and understanding}

\begin{itemize}
\item Mathematics is all in our heads.  Giving formal definitions and rigorous proofs make sure we're not just making up nonsense
\item However, humans don't think very well in this rigorous structure.  We have our own intuitive pictures
\item Most of the work of doing mathematics is translating back and forth between rigorous and intuitive modes.
\end{itemize}


\begin{block}{The \emph{Oral tradition} in mathematics}
Mathematics is written down in full rigor, but informal discussion of ``how to think about this'' or  ``what's really going on'' aren't written down
\end{block}


\begin{itemize}
\item Terry Tao, \href{https://terrytao.wordpress.com/career-advice/there’s-more-to-mathematics-than-rigour-and-proofs/}{There's more to mathematics than rigour and proofs}
\item William Thurston, \href{https://arxiv.org/abs/math/9404236}{On proof and progress in mathematics}
\end{itemize}


\end{frame}


\begin{frame}{Lectures and Notes}
\begin{itemize}
\item  Primary text: Notes by Tom Bridgeland (Rigor) on Webpage
\item  Lectures will follow notes, but from a different angle (Intuition)
\item  Slides will go online, but not what goes on board
\end{itemize}

\end{frame}



\begin{frame}[plain,c]

\begin{center}

\Huge

\usebeamercolor[fg]{frametitle}
Please \emph{Please} read the notes \\
I will be assuming you are
\end{center}

\end{frame}

\begin{frame}{Small announcements}
  \begin{block}{Office hours:}
Always by appointment is possible.
    \begin{itemize}
    \item Monday 1-2
    \item Wednesday 10-11
    \end{itemize}
\end{block}
    \begin{block}{GitHub}
      The course webpage is hosted on GitHub, a site that mostly houses software development using the Git version control system.
      \begin{itemize}
    \item This means you can find source code for all files
    \item If there's a typo / change of suggestion, you can fix it yourself and make a ``pull request''
    \end{itemize}
   Using git/github is slightly complicated and annoying, but it's a major tool used in real world, and so I encourage you to try it.
           \end{block}
\end{frame}

\begin{frame}{The first 3-4 weeks should be somewhat review}

\begin{block}{220 Syllabus}
  \url{http://maths.dept.shef.ac.uk/maths/module_info_1944.html}
\end{block}

\begin{block}{Coure notes}
  \url{https://ptwiddle.github.io/MAS439-Commutative-Algebra/MAS439Bridgeland.pdf}
  \end{block}
    
\begin{itemize}
\item You've forogtten a lot of this not having used it for two years
\item We do everything more in depth and sophisticated
\end{itemize}
\begin{block}{Talk to me!}
\alert{I AM DEPENDING ON YOU TO LET ME KNOW IF I'M GOING TOO FAST} (or too slow)
\end{block}
\end{frame}


\begin{frame}[plain,c]

\begin{center}

\Huge

\usebeamercolor[fg]{frametitle}
What's a normal subgroup?  \\ Why is a normal subgroup? \\
What's a ring?  \\ Why is a ring? \\
What's an ideal? \\
Why is an ideal? 
\end{center}

\end{frame}




\begin{frame}{Definition of a ring, ugly version}

A \emph{ring} is a set $R$ with two binary operations $+, \cdot$ satisfying:

\begin{enumerate}
\item $\forall x, y,z\in R, (x+y)+z=x+(y+z)$
\item $\exists 0_R\in R$ such that $\forall x\in R,  0_R+x=x+0_R=x$
\item $\forall x\in R, \exists {-x}\in R$ such that $x+({-x})=({-x})+x=0_R$
\item $\forall x,y \in R, x+y=y+x$
\item $\forall x, y,z\in R, (x\cdot y)\cdot z=x\cdot (y\cdot z)$
\item $\exists 1_R\in R$ such that $\forall x\in R, 1_r\cdot x=x\cdot 1_R=x$
\item $\forall x, y,z\in R$:
$$x\cdot (y+z)=x\cdot y+x\cdot z$$ 
$$(y+z)\cdot x=y\cdot x+ y\cdot z$$
\end{enumerate}

\begin{block}{What are the names of the axioms?}\end{block}

\end{frame}

\begin{frame}{Definition of a ring, take two}
A \emph{ring} is a set $R$ with two binary operations $+,\cdot$ satisfying:

\begin{enumerate}
\item $(R,+)$ is an abelian group
\item $(R,\cdot)$ is a monad
\item Multiplication $(\cdot)$ distributes over addition $(+)$
\end{enumerate}

A \emph{monad} satisfies all the axioms of a group except perhaps the existence of inverses.
\end{frame}

\begin{frame}[plain,c]

\begin{center}

\Huge

\usebeamercolor[fg]{frametitle}
Examples of rings
\end{center}

\end{frame}


\begin{frame}{How'd we do?}
\begin{enumerate}
\item The trivial ring has one element
\item The integers $\mathbb{Z}$
\item Any field $\mathbb{Q}, \mathbb{R}, \mathbb{C}, \mathbb{F}_2, \cdots$
\item ``clock arithmetic'' $\mathbb{Z}/12\mathbb{Z}$ and more generally $\mathbb{Z}/n\mathbb{Z}$
\item Polynomial rings $\mathbb{R}[x], \mathbb{\Z}[y,z]$
\item The set $M_n(\mathbb{R})$ of $n\times n$ matrices with real coefficients
\item The quaternions $\mathbb{H}$
\item The Gaussian integers $\mathbb{Z}[i]=\{z=a+bi\in \mathbb{C} | a,b\in \mathbb{Z}\}$
\item The set $\text{Fun}(\mathbb{R}, \mathbb{R})$ of all functions from $\mathbb{R}$ to itself, under pointwise addition and multiplication (e.g., $(f\cdot g)(x)=f(x)\cdot g(x)$)
\item The set $C(\mathbb{R})$ of all \emph{continuous} functions from $\mathbb{R}$ to itself
\end{enumerate}
\end{frame}


\begin{frame}{Commutative algebra and algebraic geometry}

\begin{definition}
A ring $R$ is \emph{commutative} if multiplication is commutative, i.e. $x\cdot y=y\cdot x$
\end{definition}

\begin{block}{Convention:}
Unless otherwise specified, all rings $R$ will be assumed to be commutative.
\end{block}

\begin{block}{Algebraic geometry studies the zero sets of polynomials}
$$\quad y^2-x^3=0\quad \quad y^2-x^3-x =0 \quad \quad y^2-x^3-x^2=0$$
  \end{block}
\begin{block}{Goal:}
Dictionary between commutative rings and these zero sets.
  \end{block}

\end{frame}


\begin{frame}{Types of elements}


\begin{definition}
We say $r\in R$ is a \emph{unit} if there exists an element $s\in R$ with $rs=1_R$
\end{definition}

\begin{definition}
We say that $r\in R, r\neq 0_R$ is a \emph{zero divisor}
if there exists $s\in R, s\neq 0_R$ with $rs=0_R$
\end{definition}

\begin{definition}
We say that $r\in R, r\neq 0_R$ is \emph{nilpotent}
if there exists some $n\in\mathbb{N}$ with $r^n=0_R$
\end{definition}

\begin{block}{Examples?}
 \end{block}

\end{frame}


\begin{frame}{Types of rings}

\begin{definition}
We say $R$ is \emph{field} if every nonzero element is a unit.
\end{definition}
By convention, the trivial ring is not a field.
\begin{definition}
We say $R$ is an \emph{integral domain} if it has no zero divisors.
\end{definition}

\begin{definition}
We say that $R$ is \emph{reduced} if it has no nilpotent elements.
\end{definition}

\begin{block}{Examples?}
\end{block}

\end{frame}


\end{document}
