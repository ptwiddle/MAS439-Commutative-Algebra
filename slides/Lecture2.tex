\documentclass{beamer}

\usepackage{amsmath, amssymb, hyperref, graphics}
\usepackage{mathpazo}
\usepackage[usenames, dvipsnames]{color}
\newcommand{\C}{\mathbb{C}}
\newcommand{\Z}{\mathbb{Z}}
\newcommand{\Q}{\mathbb{Q}}
\newcommand{\R}{\mathbb{R}}

\title{Commutative Algebra MAS439 \\ Lecture 2: Homomorphisms}
\author{Paul Johnson \\ \href{mailto:paul.johnson@sheffield.ac.uk}{paul.johnson@sheffield.ac.uk} \\ Hicks J06b}
\date{September 28th}

\begin{document}

\begin{frame}
\titlepage
\end{frame}

\begin{frame}{Slogan: (Homo)-morphisms preserve structure}
\begin{block}{Objects}  Often in math we study things that are sets with some extra sturcture (Groups, rings, fields, vector spaces, metric spaces, topological spaces, measure space, \dots).
\end{block}
\begin{block}{Morphisms}  In these situations, usually there is a notion of \emph{morphism} between these objects -- i.e., set maps that ``preserve the extra structure''
\end{block}
  \begin{itemize}
  \item Group homomorphisms preserve addition, units, inverses
    \item Vector space morphisms (linear maps) preserve addition and multiplication by scalars
\end{itemize}
This is the beginnings of \emph{category theory}, which we'll talk about later.
\end{frame}

\begin{frame}{What do we mean by ``preserve structure''?}

  More specifically, recall that a group $G$ has:
  \begin{itemize}
  \item an identity $e$
  \item a multiplication map $G\times G\to G: (g,h)\mapsto g\cdot h$
    \item an inverse map $G\to G:g\mapsto g^{-1}$.
  \end{itemize}
  
  \begin{definition}
    A group homomorphism $\varphi:G\to H$ is a map of sets so that
    \begin{enumerate}
    \item $\varphi(e_G)=e_H$
    \item $\varphi(g^{-1})=\varphi(g)^{-1}$
      \item $\varphi(g_1\cdot g_2)=\varphi(g_1)\cdot \varphi(g_2)$
      \end{enumerate}
\end{definition}

\alert{Warning:} For us, preserve will always be easy like this; but it can be more different. 
\end{frame}


\begin{frame}[plain,c]

\begin{center}

\Huge

\usebeamercolor[fg]{frametitle}
What is a ring homomorphism?
\end{center}

\end{frame}
\begin{frame}{Ring Homomorphisms}
  \begin{definition}
    A ring homomorphism $\varphi:R\to S$ is a function so that
    \begin{enumerate}
    \item $\varphi(0_R)=0_S$
    \item $\varphi(1_R)=1_S$
     \item $\varphi(-r)=-\varphi(r)$
     \item $\varphi(r+s)=\varphi(r)+ \varphi(s)$
       \item $\varphi(rs)=\varphi(r)\varphi(s)$
      \end{enumerate}
\end{definition}  
This is slightly more involved than the definition in the notes.  \emph{Why?}
\end{frame}

\begin{frame}{A shortcut to ring homomorphisms}
  Some of the properties in the definition are implied by other properties.
  \begin{itemize}
  \item Since $S$ has additive inverses and $\varphi$ preserves addition, it automatically preserves $0$'s:
    $$\varphi(0_R)=\varphi(0_R+0_R)=\varphi(0_r)+\varphi(0_r)$$
    and add $-\varphi(0_r)$ to both sides.
  \item Similarly, $\varphi$ must preserve additive inverses:
    $$\varphi(r)+\varphi(-r)=\varphi(r-r)=\varphi(0_R)=0_S$$
    and now add $-\varphi(r)$ to both sides.
  \end{itemize}

So we only need to check that $\varphi$ preserves addition, multiplication, and multiplicative units.
  \end{frame}


\begin{frame}[plain,c]

\begin{center}

\Huge

\usebeamercolor[fg]{frametitle}
Examples of ring homomorphisms.
\end{center}

\end{frame}

\begin{frame}{Non-examples}
  \begin{itemize}
  \item $\text{det}: M_{n\times n}(R)\to R$ is not a homomorphism: doesn't preserve addition
  \item The map $f:\Z/6\Z\to \Z/6\Z$ defined by $f([n])=[4n]$ satisfies everything but doesn't preserve the identity
    \item The map zero map $R\to S$ sending everything to $0_S$ is only a homomorphism if $S$ is the trivial ring; otherwise it doesn't preserve multiplicative identities
\end{itemize}
  \end{frame}
\begin{frame}[plain,c]

\begin{center}

\Huge

\usebeamercolor[fg]{frametitle}
Is there a ring homomorphism $\varphi:\Z\to M_{3\times 3}(\R)$? \\~\\

How many such ring homomorphisms?
\end{center}

\end{frame}

\begin{frame}{A useful lemma}

  \begin{lemma} For any ring $R$, there is a unique ring homomorphism $f:\Z\to R$.
  \end{lemma}

  To prove the lemma, we need to write down a ring homomorphism $f:\Z\to R$ to show there \emph{is} one. \\~\\

  Then, we need to prove that any other ring homomorphism \emph{has} to be the same as $f$ (uniqueness).
  \end{frame}

\begin{frame}{Isomorphisms}
Informally, we think of things as being isomorphic if they are ``the same''.  This is subtly and importantly different than being ``equal''.  

\begin{definition}
A ring homomorphism $\varphi:R\to S$ is a \emph{isomorphism} if there is another ring homomorphism $\varpsi:S\to R$ with $\varphi\circ\varpsi=\text{Id}_S, $\varpsi\circ\varphi=\text{Id}_R$.
  \end{definition} 

\begin{block}{A silly example}
  Let {\color{red} $R$} be a copy of {\color{red} $\Z$} painted {\color{red}red}.
  Let {\color{green}$S$} be a copy of {\color{green}$\Z$} painted {\color{green}green}.

  Then {\color{red}$R$} and {\color{green}$S$} are isomorphic, but they aren't equal.
  \end{block}

\end{frame}

\begin{frame}{A more serious example}
  Let $R$ be a commutative ring, and for a set $X$ recall that $\text{Fun}(X,R)$, the set of functions from $X$ to $R$, is a ring under pointwise addition and multiplication.  Let $\{x\}$ be a one element set.

  \begin{block}{$\text{Fun}(\{x\}, R)\cong R$} To prove this, we define $\varphi:\text{Fun}(\{x\}, R)\to R$ by $\varphi(f)=f(x)$. \\~\\

    For $r\in R$ let $g_r\in\text{Fun}(\{x\}, R)$ be defined by $g_r(x)=r$. Then we define $\psi:R\to\text{Fun}(\{x\}, R)$ by $\psi(r)=g_r$. \\~\\

    Then $\phi$ and $\psi$ are inverses to each other.
    \end{block}
  
Similarly, $\text{Fun}(\{x,y\},R)\cong R\times R$.
  \end{frame}

\begin{frame}{Another viewpoint on isomorphisms}
  \begin{lemma} If $\varphi:R\to S$ is a bijective homomorphism, then $\varphi$ is an isomorphism.
  \end{lemma}

  \begin{proof} Since $\varphi$ is a bijection, we know from first year that there is an inverse map $\varphi^{-1}$ of sets, we need to show that $\varphi^{-1}$ is a ring homomorphism.  \\~\\

    We need to check... (See board and/or notes)

 \end{proof}
  
\end{frame}

\begin{frame}{Nonisomorphic rings}
  Any \emph{reasonable} property of rings (i.e., defined in terms of properties of the ring structure, and not in terms of something extraneous like being {\color{green}green} or {\color{red}red}) are invariant under isomorphism.
\\~\\
So, for example, if $R$ and $S$ are isomorphic, and $R$ is an integral domain, than so is $S$.
\\~\\
To show two rings $R$ and $S$ are \emph{not} isomorphic, it is usually easiest to find something true about one ring but not the other.


\begin{lemma} None of the rings $\Z/n\Z, \Z, \Q,\R$ or $\C$ are isomorphic to each other. \end{lemma}
  \end{frame}


\begin{frame}{Sneak peek at next week}
  From a ring homomorphism $\varphi:R\to S$, we define the kernel $\ker(\phi)$ and the image $\text{Im}(\vaprhi)$ by

  $$\text{Im}(\varphi)=\{s\in S : s=\varphi(r) \text{ for some } r\in R\}$$
    $$\ker(\varphi)=\{r\in R : \varphi(r)=0_S\}$$
  

Though the kernel and the image are both subsets of a ring, it turns out they are very different types of subsets.

\begin{itemize}
\item The kernel is the prototypical (only!) example of an \emph{ideal}
  \item The image is the prototypical (only!) example of a \emph{subring}
\end{itemize}
\end{frame}


\begin{frame}{A simple use of image and kernel}


  \begin{lemma} Let $\varphi:R\to S$ a ring homomorphism.  Then

    \begin{enumerate}
    \item $\varphi$ is surjective if and only if $\text{Im}(\varphi)=S$
      \item $\varphi$ is injective if and only if $\ker(\varphi)=\{0_R\}$
      \end{enumerate}
\end{lemma}

\begin{block}{Proof}
\begin{center} ? ? ? \end{center}
\end{block}  
  \end{frame}

\end{document}
