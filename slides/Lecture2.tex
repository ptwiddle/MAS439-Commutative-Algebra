\documentclass{beamer}
\beamertemplatenavigationsymbolsempty
\usepackage{amsmath, amssymb, hyperref, graphics, tikz}

\newcommand{\C}{\mathbb{C}}
\newcommand{\Z}{\mathbb{Z}}
\newcommand{\Q}{\mathbb{Q}}
\newcommand{\R}{\mathbb{R}}
\newcommand{\N}{\mathbb{N}}

\begin{document}

\begin{frame}{Commutative Algebra Lecture 2}
\begin{block}{Last time}
\begin{itemize}
    \item Course set-up: 5 homeworks
    \item What and Why of normal subgroups (quotients!)
    \item Homomorphisms preserve structure (will review)
    \item Commutative Rings -- can add and multiply
    \item Most important example: $\Z,\Q, \R,\C, \Z/n\Z, R[x]$. Others?
    \end{itemize}
\end{block}

\begin{block}{Today}
\begin{itemize}
    \item Types of elements and rings
    \item Test clicker system
        \item Start homomorphisms (Section 4 of notes)
\end{itemize}
\end{block}
\end{frame}

\begin{frame}{Looking into the future}
\begin{itemize}
\item The last HW question depends on Section 4
\item Next session we will finish Section 4, motivate 5 and 6
\item After that, roughly one section / lecture
\end{itemize}

\begin{block}{Before next week's lecture...}
  \begin{itemize}
  \item Read Sections 2-4 of the notes
    \item Email me questions / comments you have about them
    \end{itemize}
\end{block}
\begin{block}{The first Homework is due October 18th!}
  \end{block}


\end{frame}




\begin{frame}{Basic definitions: types of elements}


\begin{definition}
We say $r\in R$ is a \emph{unit} if there exists an element $s\in R$ with $rs=1_R$
\end{definition}

\begin{definition}
We say that $r\in R$ is a \emph{zero divisor} if there exists $s\in R, s\neq 0_R$ with $rs=0_R$
\end{definition}

\begin{definition}
We say that $r\in R$ is \emph{nilpotent} if there exists some $n\in\mathbb{N}$ with $r^n=0_R$
\end{definition}

\begin{block}{Examples!}
 \end{block}
\end{frame}


\begin{frame}{Basic definitions: Types of rings}

\begin{definition}
We say $R$ is \emph{field} if every nonzero element is a unit.
\end{definition}
By convention, the trivial ring is not a field.
\begin{definition}
We say $R$ is an \emph{integral domain} if it has no zero divisors.
\end{definition}

\begin{definition}
We say that $R$ is \emph{reduced} if it has no nilpotent elements.
\end{definition}

\begin{block}{Examples!}
\end{block}
\begin{theorem}
$R$ a field $\implies$ $R$ an integral domain $\implies$ $R$ is reduced
\end{theorem}

\end{frame}


\begin{frame}[plain, c]

\begin{center}

\Huge

\usebeamercolor[fg]{frametitle}
Clicker session: ttpoll.eu
\end{center}

\end{frame}



\begin{frame}{Review: (Homo)-morphisms preserve structure}
\begin{block}{Objects}  Often in math we study things that are sets with some extra sturcture (Groups, rings, fields, vector spaces, metric spaces, topological spaces, measure space, \dots).
\end{block}
\begin{block}{Maps or Morphisms}  In these situations, usually there is a notion of \emph{map} or \emph{morphism} between these objects -- these are functions that ``preserve the extra structure''
\end{block}
  \begin{itemize}
  \item Group homomorphisms preserve addition, units, inverses
    \item Vector space morphisms (linear maps) preserve addition and multiplication by scalars
\end{itemize}

\end{frame}

\begin{frame}{What do we mean by ``preserve structure''?}

  More specifically, recall that a group $G$ has:
  \begin{itemize}
  \item an identity $e$
  \item a multiplication map $G\times G\to G: (g,h)\mapsto g\cdot h$
    \item an inverse map $G\to G:g\mapsto g^{-1}$.
  \end{itemize}
  
  \begin{definition}
    A group homomorphism $\varphi:G\to H$ is a map of sets so that
    \begin{enumerate}
    \item $\varphi(e_G)=e_H$
    \item $\varphi(g^{-1})=\varphi(g)^{-1}$
      \item $\varphi(g_1\cdot g_2)=\varphi(g_1)\cdot \varphi(g_2)$
      \end{enumerate}
\end{definition}

\alert{Warning:} For us, ``preserve the structure'' doesn't have to be this striaghtforward.
\end{frame}


\begin{frame}[plain,c]

\begin{center}

\Huge

\usebeamercolor[fg]{frametitle}
What is a ring homomorphism?
\end{center}

\end{frame}



\begin{frame}{Ring Homomorphisms}
  \begin{definition}
    A ring homomorphism $\varphi:R\to S$ is a function so that
    \begin{enumerate}
    \item $\varphi(0_R)=0_S$
    \item $\varphi(1_R)=1_S$
     \item $\varphi(-r)=-\varphi(r)$
     \item $\varphi(r+s)=\varphi(r)+ \varphi(s)$
       \item $\varphi(rs)=\varphi(r)\varphi(s)$
      \end{enumerate}
\end{definition}  
This is slightly more involved than the definition in the notes, because some of these properties follow from others...
\begin{block}{Which?}
\end{block}
\end{frame}

\begin{frame}[plain, c]

\begin{center}

\Huge

\usebeamercolor[fg]{frametitle}
Examples of ring homomorphisms!
\end{center}

\end{frame}



\end{document}
