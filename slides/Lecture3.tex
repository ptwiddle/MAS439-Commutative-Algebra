\documentclass{beamer}

\usepackage{amsmath, amssymb, hyperref, graphics}
\usepackage{mathpazo}
\newcommand{\C}{\mathbb{C}}
\newcommand{\Z}{\mathbb{Z}}
\newcommand{\Q}{\mathbb{Q}}
\newcommand{\R}{\mathbb{R}}

\title{Commutative Algebra MAS439 \\ Lecture 3: Subrings}
\author{Paul Johnson \\ \href{mailto:paul.johnson@sheffield.ac.uk}{paul.johnson@sheffield.ac.uk} \\ Hicks J06b}
\date{October 4th}

\begin{document}

\begin{frame}
\titlepage
\end{frame}

\begin{frame}{Plan: slow down a little}

  \begin{block}{Last week - Didn't finish}
    \begin{itemize}
\item Course policies + philosophy
    \item  Sections 2-4: Rings, examples, homomorphisms
\end{itemize}
  \end{block}
  \begin{block}{Today}
\begin{itemize}
\item    Finish Section 4: Image and kernel of homomorphisms
\item    Cover Section 5: Subrings
\end{itemize}
\end{block}
\begin{block}{Tomorrow}
  \begin{itemize}
    \item Cover Section 6: Ideals
\end{itemize}
  \end{block}
\end{frame}



\begin{frame}{Kernels and Images, ideals and subrings}
  From a ring homomorphism $\varphi:R\to S$, we define the kernel $\ker(\phi)$ and the image $\text{Im}(\varphi)$ in the same way we did for linear maps of vector spaces:

  $$\text{Im}(\varphi)=\{s\in S : s=\varphi(r) \text{ for some } r\in R\}$$
    $$\ker(\varphi)=\{r\in R : \varphi(r)=0_S\}$$

Though the kernel and the image are both subsets of a ring, it turns out they are very different types of subsets.

\begin{itemize}
\item The kernel is the prototypical (only!) example of an \emph{ideal}
  \item The image is the prototypical (only!) example of a \emph{subring}
\end{itemize}
\end{frame}

\begin{frame}{A simple use of image and kernel}

  \begin{lemma} Let $\varphi:R\to S$ a ring homomorphism.  Then

    \begin{enumerate}
    \item $\varphi$ is surjective if and only if $\text{Im}(\varphi)=S$
      \item $\varphi$ is injective if and only if $\ker(\varphi)=\{0_R\}$
      \end{enumerate}
\end{lemma}

\begin{block}{Proof}
\begin{center} ? ? ? \end{center}
\end{block}  
  \end{frame}


\begin{frame}{Definition of a subring}

  Let $R$ be a ring, and let $S\subset R$ be a subset.
  
  \begin{block}{Idea}
    We say $S$ is a subring of $R$ if it is a ring, and all its structure comes from $R$.
  \end{block}

  \begin{definition}
We say $S\subset R$ is a subring if:
    \begin{itemize}
\item $S$ is closed under addition and multiplication:
$$r,s\in S \text{ implies } r+s, r\cdot s\in S$$
\item $S$ is closed under additive inverses: $r\in S \text{ implies } -r\in S$.
\item $S$ contains the identity: $1_R\in S$
\end{itemize}
\end{definition}

  \begin{lemma} A subring $S$ is a ring.
    \end{lemma}
  

\end{frame}


\begin{frame}{First examples of subrings}

\begin{itemize}
\item $\mathbb{Z}\subset\mathbb{Q}\subset \mathbb{R}\subset\mathbb{C}\subset\mathbb{H}$ is a chain of subrings.
\item if $R$ any ring, $R\subset R[x]\subset R[x,y]\subset R[x,y,z]$ is a chain of subrings.
\end{itemize}
\end{frame}

\begin{frame}{Another chain of subrings}

$$\mathbb{R}\subset \mathbb{R}[x]\subset C^\infty(\mathbb{R},\mathbb{R})\subset C(\mathbb{R},\mathbb{R}) \subset \textrm{Fun}(\mathbb{R},\mathbb{R})$$

Where, working backwards:
\begin{itemize}
\item $\textrm{Fun}(\mathbb{R},\mathbb{R})$ is the space of all functions from $\mathbb{R}$ to $\mathbb{R}$
\item $C(\mathbb{R},\mathbb{R})$ are the continuous functions
\item $C^\infty(\mathbb{R},\mathbb{R})$ are the \emph{smooth} (infinitely differentiable) functions
\item $\mathbb{R}[x]$ are the polynomial functions
\item We view $\mathbb{R}$ as the space of constant functions 
\end{itemize}

\end{frame}


\begin{frame}{Non-examples of subrings}

\begin{itemize}
\item $\mathbb{N}\subset\mathbb{Z}$ 
\item Let $\mathcal{K}$ be the set of continuous functions from $\mathbb{R}$ to itself with bounded support.  That is,
$$f\in \mathcal{K} \iff \exists M \textrm{ s.t. } |x|>M\implies  f(x)=0 $$
\item Let $R=\mathbb{Z}\times \mathbb{Z}$, and let $S=\{(x,0)\in R | x\in \mathbb{Z}\}$.  
\item $\{0,2,4\}\subset\Z/6\Z$
\end{itemize}
\end{frame}

\begin{frame}{The image of a homomorphism is a subring}

\begin{lemma} Let $\varphi:R\to S$ be a homomorphism.  Then $\textrm{Im}(\varphi)\subset S$ is a subring.
\end{lemma}

\begin{proof}
We need to check $\textrm{Im}(\varphi)$ is closed under addition and multiplication and contains $1_S$.
\end{proof}

\end{frame}


\begin{frame}[plain,c]

\begin{center}

\Huge

\usebeamercolor[fg]{frametitle}
Generating subrings
\end{center}

\end{frame}


\begin{frame}{Motivation for generators from Group theory}

When working with groups, we often write groups down in terms of generators and relations.

\begin{block}{Generators are easy}
To say a group $G$ is \emph{generated} by a set of elements $E$, means that we can get $G$ by ``mashing together'' the elements of $E$ in all possible ways. More formally, 
$$G=\{g_1\cdot g_2\cdots g_n | g_i \text{ or } g_i^{-1}\in E\}$$
\end{block}


\begin{block}{Relations are harder}
Typically there will be many different ways to write the same element in $G$ as a product of things in $E$; recording how is called relations.
\end{block}
\end{frame}

\begin{frame}{Reminder example?  Okay if it's new to you }
\begin{example}
The dihedral group $D_{8}$ is the symmetries of the square.  It is often written as

$$D_8=\langle r,f | r^4=1, f^2=1, rf=fr^{-1}\rangle $$

Meaning that the group $D_8$ is \emph{generated} by two elements, $r$ and $f$, satisfing the \emph{relations} $r^4=1, f^2=1$ and $rf=fr^{-1}$.
\end{example}

\begin{block}{We'll want a way to write down commutative rings in the same way}
\end{block}


\end{frame}


\begin{frame}{Preview of rings from generators and relations}

We will revist these examples further after we have developed ideals and quotient rings -- you can think of these as the machinery that will let us impose relations on our generators.

\begin{example}[Gaussian integers]
The Gaussian integers are written $\mathbb{Z}[i]$; they're generated by an element $i$ satisfying $i^4=1$.
\end{example}

\begin{example}[Field with 4 element]
The field $\mathbb{F}_4$ of four elements can be written $\mathbb{F}_2[x]/(x^2+x+1)$ -- to get $\mathbb{F}_4$, we add an element $x$ that satisfies the relationship $x^2+x+1=0$.
\end{example}

\end{frame}


\begin{frame}{Idea of generating set}

The subring generated by elements in a set $T$ will again be ``what you get when you mash together everything in $I$ in all possibly ways'', but this is a bit inelegant and not what we will take to be the \emph{definition}.


\begin{block}{Attempted definition} Let $T\subset R$ be any subset of a ring.  The \emph{subring generated by $T$}, denoted $\langle T\rangle$, \emph{should be} the smallest subring of $R$ containing $T$.
\end{block}

This is not a good formal definition -- what does ``smallest'' mean?  Why is there a smallest subgring containing $T$? 


\end{frame}

\begin{frame}{Intersections of subrings are subrings}


\begin{lemma} Let $R$ be a ring and $I$ be any index set. For each $i\in I$, let $S_i$ be a subring of $R$.  Then
$$S=\bigcap_{i\in I} S_i$$
is a subring of $R$.

\end{lemma}
\begin{proof} 
\begin{center} ? ? ? ? ? \end{center}
\end{proof}

\end{frame}

\begin{frame}{The elegant definition of $\langle T\rangle$}

\begin{definition} Let $T\subset R$ be any subset.  The \emph{subring generated by $T$}, denoted $\langle T\rangle$, is the intersection of all subrings of $R$ that contain $T$.
\end{definition}

\begin{block}{This agrees with our intuitive ``definition''}
$\langle T\rangle $ is the smallest subring containing $T$ in the following sense: if $S$ is any subring with $T\subset S\subset R$, then by definition $\langle T\rangle \subset S$.
\end{block}

\begin{block}{But it's all a bit airy-fairy}
The definition is elegant, and can be good for proving things, but it doesn't tell us what, say $\langle \pi, i \rangle \subset \mathbb{C}$ actually looks like.  Back to ``mashing things up''...
\end{block}


\end{frame}


\begin{frame}{What \emph{has} to be in $\langle \pi, i \rangle $?  Start mashing!}

Rings are a bit more complicated because there are two ways we can mash the elements of $T$ -- addition and multiplication.

\begin{itemize}
\item $1, \pi, i$
\item Sums of those; say, $5+\pi, 7i$
\item Negatives of those, say $-7i$ 
\item Products of those, say $(5+\pi)^4 i^3$
\item Sums of what we have so far, say $(5+\pi)^4i^3-7i+3\pi^2$
\item $\cdots$
\end{itemize}
leading to things like:
$$\Big(\big((5+\pi)^4i^3-7i+3\pi^2\big)\cdot (-2+\pi i)+\pi^3-i\Big)^{27}-5\pi^3i$$

Of course, could expand that out into just sums of terms like $\pm\pi^m i^m$...

\end{frame}

\begin{frame}{Formalizing our insight}

\begin{definition} Let $T\subset R$ be any subset.  Then a \emph{monomial in $T$} is a (possibly empty) product $\prod_{i=1}^n t_i$ of elements $t_i\in T$.  We use $M_T$ to denote the set of all monomials in $T$.
\end{definition}

\begin{block}{Note:}
The empty product is the identity $1_R$, and so $1_R\in M_T$.
\end{block}

\begin{block}{Our insight:}

From the ``mashing'' point of view $\langle T\rangle$ should be all $\mathbb{Z}$-linear combination of monomials.  

\end{block}

\end{frame}


\begin{frame}{The elegant and ``mashing'' definitions agree}

\begin{lemma} $\langle T\rangle=X_T$, where $X_T$ consists of those elements of $R$ that are finite sums of monomials in $T$ or their negatives.  That is: 
$$X_T=\left\{\sum_{k=0}^n \pm m_k \Big| m_k\in M_T\right\}$$

\end{lemma}

\begin{proof} 
\begin{itemize}
\item $X_T\subset \langle T\rangle$? 
\item $\langle T\rangle \subset X_T$?
\end{itemize}
\end{proof}

\end{frame}

\begin{frame}{Example: The Gaussian integers}

\begin{block}{What's $\langle i\rangle\subset\C$?}
  \begin{itemize}
  \item What's the set of monomials?
  \item But can we simplify even more?
  \end{itemize}
  
  
\end{block}




\end{frame}



\begin{frame}{Generating sets for rings}

\begin{definition}
We say that a ring $R$ is \emph{generated by} a subset $T$ if $R=\langle T\rangle$.  We say that $R$ is \emph{finitely generated} if $R$ is generated by a finite set.
\end{definition}

\end{frame}


\begin{frame}{Examples of generating sets}
\begin{itemize}
\item $\mathbb{Z}=\langle\emptyset\rangle $
\item $\mathbb{Z}/n\mathbb{Z}=\langle \emptyset \rangle $
\item $\mathbb{Z}[x]=\langle x \rangle=\langle 1+x\rangle $
\item $\mathbb{Z}[i]=\langle i \rangle$
\end{itemize}



\end{frame}


\begin{frame}{Some of your best friends are not finitely generated}

\begin{itemize} 
\item The rationals $\mathbb{Q}$ are not finitely generated: any finite subset of rational numbers has only a finite number of primes appearing in their denominator.  
\item The real and complex numbers are uncountably; a finitely generated ring is countable
\end{itemize}
\end{frame}

\begin{frame}{A non-finitely generated subring of a finitely generated ring}


We've seen that $\mathbb{Z}[x]=\langle x\rangle$ and so is finitely generated.

$$S=\left\{ a_0+2a_1x+\cdots +2a_nx^n\right\}$$
that is, $S$ consists of polynomials all of whose coefficients, except possibly the constant term, are even.   



\begin{block}{Challenge:}
Show that $S$ is a subring of $\mathbb{Z}[x]$ (easy), but that $S$ is not finitely generated (harder).
\end{block}

\end{frame}

\begin{frame}[plain,c]

\begin{center}

\Huge

\usebeamercolor[fg]{frametitle}
Material on isomorphisms of rings
\end{center}

\end{frame}



\begin{frame}{Isomorphisms}
Informally, we think of things as being isomorphic if they are ``the same''.  This is subtly and importantly different than being ``equal''.  

\begin{definition}
A ring homomorphism $\varphi:R\to S$ is a \emph{isomorphism} if there is another ring homomorphism $\psi:S\to R$ with $$\varphi\circ\psi=\text{Id}_S, \quad\psi\circ\varphi=\text{Id}_R$$
  \end{definition} 

\begin{block}{A silly example}
A green copy of $\Z$ and a red copy of $\Z$ are isomorphic, but they aren't equal.
\end{block}

\end{frame}

\begin{frame}{A nontrivial example}
  \begin{lemma} Let $X=\{x_1,\dots, x_n\}$ be a finite set with $n$ elements, and let $R$ be a ring.  Then

    $$\text{Fun}(X,R)\cong R^n:=\underbrace{R\times R\times\cdots\times R}_{\text{$n$ times}}$$
    \end{lemma}
  \begin{block}{Proof}
    \begin{itemize}
\item    Define $\varphi:\text{Fun}(X,R)\to R^n$ by $\varphi(f)=(f(x_1),f(x_2),\dots, f(x_n))$,
\item    Define $\psi:R^n\to\text{Fun}(X,R)$ by $[\psi(r_1,\dots, r_n)](x_i)=r_i$.
\item Check a buncha stuff.
    \end{itemize}
   \end{block}
 
\end{frame}

\begin{frame}{Another viewpoint on isomorphisms}
  \begin{lemma} If $\varphi:R\to S$ is a bijective homomorphism, then $\varphi$ is an isomorphism.
  \end{lemma}

  \begin{proof} Since $\varphi$ is a bijection, we know from first year that there is an inverse map $\varphi^{-1}$ of sets, we need to show that $\varphi^{-1}$ is a ring homomorphism.  \\~\\

    We need to check... (See board and/or notes)\end{proof}

In notes, this is taken as the \emph{definition} of isomorphic rings, but the definition we gave is the \emph{right} one because it generalizes.  It is NOT true that if $f:X\to Y$ is a bijective continuous map of topological spaces, then $X\cong Y$.
  
\end{frame}

\begin{frame}{Nonisomorphic rings}
  Any \emph{reasonable} property of rings (i.e., defined in terms of properties of the ring structure, and not in terms of something extraneous like being {\color{green}green} or {\color{red}red}) are invariant under isomorphism.
\\~\\
So, for example, if $R$ and $S$ are isomorphic, and $R$ is an integral domain, than so is $S$.
\\~\\
To show two rings $R$ and $S$ are \emph{not} isomorphic, it is usually easiest to find something true about one ring but not the other.


\begin{lemma} None of the rings $\Z/n\Z, \Z, \Q,\R$ or $\C$ are isomorphic to each other. \end{lemma}
\begin{block}{Proof:}
  \begin{center} ? ? ? \end{center}
  \end{block}

\end{frame}



\end{document}
