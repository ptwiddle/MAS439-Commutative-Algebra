\documentclass{beamer}

\usepackage{amsmath, amssymb, hyperref, graphics, wasysym}
\usepackage{mathpazo}

\newcommand{\C}{\mathbb{C}}
\newcommand{\Z}{\mathbb{Z}}


\title{MAS439 Lecture 5 \\ Ideals}

\date{October 12th}

\begin{document}

\begin{frame}
\titlepage
\end{frame}


\begin{frame}{Motivation for ideals: kernels}

Let $\varphi:R\to S$ be a ring homomorphism, we have $\textrm{Im}(\varphi)\subset S$ and $\ker(\varphi)\subset R$.

\begin{block}{Our study of subrings was motivated by $\textrm{Im}(\varphi)$:}
\begin{itemize}
\item The image of $\varphi$ is a subring
\item Every subring $S\subset R$ is $\textrm{Im}(i)$ for $i:S\to R)$
\end{itemize}
\end{block}

\begin{block}{Study of ideals are motivated by $\ker(\varphi)$}
\begin{itemize}
\item Last time: $\ker(\varphi)$ is an ideal
\item Next lecture: Every ideal $I$ is $\ker(\varphi)$ for $\varphi:R\to R/I$
\end{itemize}
\end{block}

\begin{block}{Next week: Isomorphism theorem}
Every homomorphism $\varphi:R\to S$ can be factored as a surjective map $\varphi:R\to R/\ker(\varphi)$ and the inclusion $i:R/\ker(\varphi)\to S$
\end{block}
\end{frame}


\begin{frame}{Recall: definition of ideals}
\begin{definition}
An ideal $I\subset R$ is a nonempty subset that is

\begin{itemize}
\item Closed under addition: $I+I\subset I$
\item Closed under multiplication by elements of $R$: $rI\subset I$.
\end{itemize}
\end{definition}

\begin{block}{Remarks:}
\begin{itemize}
\item Ideals are additive subgroups.  Closed under addition, and if $r\in I$, then $(-1)\cdot r=-r\in I$.
\item If $1\in I$, then $I=R$, since $r=r\cdot 1\in I$.
\end{itemize}

\end{block}

\end{frame}


\begin{frame}{First examples of ideals}

\begin{itemize}
\item In any ring $R$, the set $\{0_R\}$ is an ideal
\item In any ring $R$, the set $R$ is an ideal
\item The set $n\mathbb{Z}=\{k\in \mathbb{Z} | n \textrm{ divides } k \}$
\item The even numbers in $\mathbb{Z}/12\mathbb{Z}$ form an ideal
\item The set $\{0\}\times R\subset S\times R$ is an ideal in $S\times R$
\item The set $I\subset R[x]$ of polynomials whose constant term is 0 is an idea
\end{itemize}
\begin{block}{Which of these are kernels of homomorphisms?}
  \end{block}

\end{frame}

\begin{frame}{Another set of examples}
Let $X$ be any set, $R$ any ring, $S\in X$ any subset.  The set of all functions $f:X\to R$ vanishing at all $x\in S$ is an ideal.
$$I(S)=\{f\in Fun(X,R): f(x)=0 \text{ for all } x\in S\}$$

\begin{block}{Is $I(S)$ the kernel of a homomorphism?}
\end{block}

\end{frame}


\begin{frame}{The ideals of $\mathbb{Z}$}

\begin{lemma} Any ideal of $\mathbb{Z}$ is of the form $n\mathbb{Z}$ for some $n$.
\end{lemma}

\begin{proof}
Let $I\subset \mathbb{Z}$ be an ideal; assume $I\neq \{0\}$, and let $n$ be the smallest nonnegative element of $I$.  We claim that $I=n\mathbb{Z}$.

We have $n\mathbb{Z}\subset I$.  

To prove that $I\subset n\mathbb{Z}$, suppose not, and let $x\ in I, x\notin n\mathbb{Z}$.  We have $x=qn+r$, for some $0\leq r<n$.  $r\neq 0$, as $x\notin n\mathbb{Z}$.  But $r=x-qn\in I$, which contradicts $n$ being the smallest positive element of $I$.

\end{proof}
\end{frame}




\begin{frame}{Generating Ideals}

Parallel to what we did yesterday with subrings, we now want to discuss generating ideals.

\begin{block}{Intuitive idea of generating an ideal}
Let $T\subset R$ a subset.  The \emph{ideal generated by $T$}, denoted $(T)$, should be the smallest ideal of $R$ containing $T$.
\end{block}

It's not clear that this is defined, and it's not how we normally think of generators as ``mashing things together''.  Hopefully you can see what's coming next..
\end{frame}

\begin{frame}{Intersections of ideals are ideals}
\begin{lemma} Suppose that $I_j$, for $j\in J$ an index set, is set of ideals of $R$.  Then $\bigcap_{j\in J} I_j$ is an ideal.
\end{lemma}

\begin{proof} 
 \begin{center} ? ? ? ? ? \end{center}
\end{proof}


\end{frame}


\begin{frame}{The elegant definition}

\begin{definition}
The ideal generated by $T$, denoted $(T)$, is the intersection of all ideals of $R$ containing $T$.
\end{definition}

Again, this is a nice definition and easy to prove things with, but doesn't tell us what elements of $(T)$ look like.  So, we now find another description of $(T)$ in terms of ``mashing'' elements of $T$ and $R$ together.

\end{frame}


\begin{frame}{This description is easy for ideals than rings}


\begin{lemma}
The ideal $(T)$ consists of all finite sums of the form
$$r=r_1\cdot t_1+\cdots+r_kt_k, k\geq 0; r_j\in R, t_j\in T$$
\end{lemma}
\begin{proof}
Let $X$ be the set of such elements.  

\begin{itemize}
\item $X\subset (T)$: since the $t_i\in (T)$, and $T$ is closed under addition and multiplication by elements of $R$.
\item $(T)\subset X$: clearly $X$ contains $T$, so by the definition of $(T)$ it is enough to show that $X$ is an ideal.  $X$ is nonempty, is closed under addition, and by the distributive property is closed under multiplication by elements of $R$.
\end{itemize}
\end{proof}

\end{frame}


\begin{frame}{Generating subrings vs. generating ideals}
Consider the subset $T=\{2,x\}\subset\mathbb{Z}[x]$ 
\begin{itemize}
\item The subring generated by $T$, $\langle T\rangle$, consists of all of $\mathbb{Z}[x]$; in fact, we already have $\langle x\rangle=\mathbb{X}[x]$, and including the two does nothing.
\item The ideal generated by $T$, $(T)=(2,x)$ consists of all polynomials with even constant term, i.e., those of the form
$$2a_0+a_x+\cdots+a_nx^n, a_i\in\mathbb{Z}$$
Note that here the inclusion of the two is essential; the ideal generated by just $x$, $(x)$, consists of all polynomials with \emph{vanishing} constant term.

\end{itemize}



\end{frame}




\begin{frame}{Two Analogies}

\begin{enumerate}
\item Yesterday, we saw the subring generated by $T$ looked like \emph{polynomials} in $T, \mathbb{Z}[T]$.

\item In contrast, the ideal generated by $T$ looks like a \emph{vector space} over $R$ with $T$ as a generating set.
\end{enumerate}

Both of these rough observations will be made more formal.   The observations about subrings will be developed this semester when we talk about polynomial rings and quotient rings.  Next semester, the similarity of ideals and vector spaces will be fit into a larger concept of \emph{modules}


\end{frame}

\begin{frame}{Generating ideals}

\begin{definition} If $I=(T)$, we say that $T$ generates $I$.  We say that $I$ is \emph{finitely generated} if $I=(T)$ for $T$ a finite set.  If $T=\{t_1,\dots, t_n\}$ is a finite set, we write $(t_1,\dots, t_n)$ instead of $(T)$.
\end{definition}

\begin{definition} An ideal generated by a single element, 
$$I=(t)=\{r\cdot t: r\in R\}$$
is called \emph{principal}.  If every ideal in $R$ is principal, we call $R$ an \emph{principal ideal domain}.
\end{definition}

\end{frame}


\begin{frame}{Are you a principal ideal domain?}

\begin{block}{Examples of principal ideal domains:}
\begin{itemize}
\item We saw the $\mathbb{Z}$ is a principal ideal domain
\item Any quotient ring $\mathbb{Z}/n\mathbb{Z}$ is a P.I.D.
\item For any field $k$, $k[x]$ is a principal ideal domain.
\end{itemize}
For last two, why?
\end{block}

\begin{block}{Rings that aren't principal ideal domains:}
\begin{itemize}
\item $\mathbb{Z}[x]$
\item For $k$ a field, $k[x,y]$
\end{itemize}
Why?
\end{block}


\end{frame}


\end{document}
