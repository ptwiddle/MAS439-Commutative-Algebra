\documentclass{beamer}

\usepackage{amsmath, amssymb, hyperref, graphics, wasysym}
\usepackage{mathpazo}

\newcommand{\C}{\mathbb{C}}
\newcommand{\Z}{\mathbb{Z}}
\newcommand{\R}{\mathbb{R}}

\title{MAS439 Lecture 6 \\ Examples of Quotient Rings \\ What's the Universal Property all about?}

\date{October 12th}

\begin{document}

\begin{frame}
\titlepage
\end{frame}

\begin{frame}{Last time, we introduced the quotient ring $R/I$}
  \begin{definition} Let $R$ be a ring, and $I$ an ideal.  Then the ring $R/I$ is the set of equivalence classes of elements of $R$, where $r\sim s$ if $r-s\in I$.

    Addition and multiplication are given by adding and multiplying representatives:
    $$[r]+[s]=[r+s]$$
    $$[r]\cdot [s]=[r\cdot s]$$
    $$1_{R/I}=[1]$$
  \end{definition}

  \begin{lemma}The map $p:R\to R/I$ defined by $p(r)=[r]$ is a homomorphism.
  \end{lemma}
  \end{frame}


\begin{frame}[plain,c]

\begin{center}

\Huge

\usebeamercolor[fg]{frametitle}
Examples
\end{center}

\end{frame}

\begin{frame}{The problem with this definition:}
  To talk about what the elements are, we need to understand what the equivalence classes are.

  \begin{block}{Usually we want to pick a unique representative from each class}
    This is exactly like thinking:
    $$\Z/n=\{0,1,\dots, n-1\}$$
    instead of the strict definition:
    $$\Z/n=\Big\{\{a+n\Z\} : a\in \Z\Big\}$$
    \end{block}
  \begin{block}{The division algorithm is a good way to do this}
    \end{block}


  \end{frame}

\begin{frame}{Example: $\C\cong\R[x]/(x^2+1)$}

\begin{block}{The division algorithm gives unique representatives}

Any polynomial $p(x)$ can be written uniquely as

 $$p(x)=(x^2+1)q(x)+bx+a$$. 

This means that $[p(x)]=[bx+a]$, so every class can be represented by a linear polynomial; furthermore, this representation is unique.
\end{block}

It's clear $[a+bx]+[c+dx]=[a+c+(b+d)x]$.
\end{frame}

\begin{frame}{Example: $\mathbb{C}=\R[x]/(x^2+1)$}

\begin{block}{Multiplication of representatives}
$$[a+bx]\cdot [c+dx]=[ac+(ad+bc)x+bdx^2]$$

But this isn't linear;  we need to get rid of the $x^2$ term.  Note that $bdx^2=bd(x^2+1)-bd$, and so $[bdx^2]=[-bd]$.

Thus, we see
$$[a+bx]\cdot [c+dx]=[ac-bd+(ad+bc)x]$$
which, if we replace $x$ with $i$, is exactly the formula for multiplying complex numbers.
\end{block}


\end{frame}


\begin{frame}{Example: $\R[x]/(x^2)$}

First, we have to understand it as a set -- we want to give a \emph{unique} name to each element of $R/I$.  This is usually done by picking a representative from each coset in some systematic way.

$I$ consists of linear combinations of monomials of degree 2 or bigger.  So every equivalence class contains exactly one linear term $a+bx$.  We see that 

$$[a+bx]\cdot [c+dx]=[ac+adx+bcx+adx^2]=[ac+(ad+bc)x]$$


\end{frame}




\begin{frame}{Constructing $\mathbb{F}_4$}

We claim that $R=\mathbb{F}_2[x]/(x^2+x+1)$ is a field with 4 elements.  

Exactly as in the last two examples, the division algorithm gives every equivalence class has a unique linear representative $a+bx$; now $a,b\in\mathbb{F}_2$, so there are indeed four elements.

We check:
$$[x]\cdot [x+1]=[x^2+x]=[1]$$
So every nonzero element has an inverse, and so $R$ is a field.
\end{frame}

\begin{frame}{A case where the division algorithm doesn't hold:}
\begin{block}{Consider the ring $\Z[x]/(2x-1)$:}
Can't divide $x$ by $2x-1$ and get a polynomial of lower degree.  
\end{block}

\begin{block}{What is this ring, intuitively?}
  Since we're setting $2x-1=0$, then $x$ ``should be'' $1/2$.  So, we've taken the integers and added $1/2$.
  \end{block}

\begin{block}{How to make this intuition formal?}
To really understand the ring $\Z[x]/(2x-1)$, will find two different systems ofunique representatives for the equivalence classes.
\end{block}
  
  \end{frame}


\begin{frame}{Method 1: Muddle along with division algorithm}
  \begin{lemma}
    For any polynomial $p(x)\in\Z[x]$, there is a unique polynomials $q(x)$ and a unique integer $r$, so that
    \begin{itemize}
    \item $p(x)=q(x)(2x-1)+r$
    \item $q(x)=\sum a_n x^n$ with $a_i\in \{0,1\}$
    \end{itemize}
  \end{lemma}

  \begin{proof}
    \begin{itemize}
    \item Existence
    \item Uniqueness
\end{itemize}
    \end{proof}

 Every number in $\Z[1/2]$ has a unique terminating binary expansion.


  \end{frame}

\begin{frame}{Method 2: Divide ``backwards''}

  \begin{lemma} For any $p(x)\in\Z[x]$, there is a unique $q(x)\in\Z[x]$, $n\geq 0$, and $a\in\Z$, so that
    $$p(x)=q(x)(1-2x)+ax^n$$
and $a$ is odd if $n>0$.  
  \end{lemma}
  \begin{proof}
    \begin{itemize}
    \item Existence: divide backwards as power series to get remainder $ax^m$, if even, backtrack;
    \item Uniqueness
    \end{itemize}
  \end{proof}

  This is writing an element of $\Z[1/2]$ as $a/2^n$, and then if $a$ is even we can cancel some powers of 2.
  

  \end{frame}

\begin{frame}{

\end{document}
