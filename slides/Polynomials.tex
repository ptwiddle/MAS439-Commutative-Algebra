\documentclass{beamer}
\beamertemplatenavigationsymbolsempty
\usepackage{amsmath, amssymb, hyperref, graphics, wasysym}
\usepackage{mathpazo}

\newcommand{\C}{\mathbb{C}}
\newcommand{\Z}{\mathbb{Z}}
\newcommand{\R}{\mathbb{R}}


\begin{document}



\begin{frame}{Section 11: Polynomial algebras}
  Going to do this section a bit fast, as there's not that much there.

It formally introduce $k[x_1,\dots, x_n]$, and answers the question:
  \begin{block}{Why are polynomial rings and their quotients important?}
First answer: $k[x]$ satisfies a universal property. \\
    Because of universal property:

      \begin{center}
      \begin{huge}
\usebeamercolor[fg]{frametitle}
        $R$ is a finitely generated $k$-algebra \\
        $\iff$ \\
        $R\cong k[x_1,\dots, x_n]/I$ 
        \end{huge}
      \end{center}
    \end{block}


\end{frame}

\begin{frame}{Universal Property of $k[x]$}
  \begin{lemma}$k[x]$  and $x$ satisfies the following universal property: for any $k$-algebra $S$ and any element $s\in S$, there is a unique $k$-algebra homomorphism $\varphi_s:k[x]\to S$ such that $\varphi_s(x)=s$.
  \end{lemma}
  \begin{proof}
    Plug $s$ in for $x$.
  \end{proof}
  \begin{lemma} If $R,r$ is any $k$-algebra satisfying the universal property of $k[x]$, then there is a unique isomorphism between $R$ and $k[x]$ identifying $r$ with $x$.
  \end{lemma}
Similarly, $k$-algebra homomorphisms $k[x_1,\dots, x_n]$ to $R$ are the same thing as $n$-tuples of elements $r_1,\dots, r_n\in R$.
\end{frame}

  \begin{frame}{Finitely generated = Quotient of Polynomial Algebra}
\begin{block}{Finitely Generated $\implies$ quotient}
    Suppose $R$ is generated by $r_1,\dots, r_n$.  
    \begin{itemize}
\item The homomorphism $\varphi:k[x_1,\dots, x_n]\to R$ sending $x_i$ to $r_i$ is surjective.
\item By first isomorphism theorem $R[x_1,\dots, x_n]/\ker(\varphi)\cong R$.
    \end{itemize}
\end{block}
\begin{block}{Quotient $\implies$ finitely generated}
In the other direction, if $R=k[x_1,\dots, x_n]/I$ for some ideal $I$, then $R$ is generated by $[x_1],\dots, [x_n]$.
\end{block}
So it might \emph{seem} like it's restrictive to study $k[x_1,\dots, x_n]/I$, but we're really studying finitely generated $k$-algebras.
\begin{block}{Can we have infinitely many relations?}
I.e., does $I$ need to be finitely generated?
\end{block}
  \end{frame}
  \begin{frame}{Section 12: Noetherian rings}
    \begin{definition}A ring $R$ is \emph{Noetherian} if it satisfies the \emph{Ascending Chain Condition}, or A.C.C., namely, if every ascending chain of ideals $$I_1\subseteq I_2\subseteq I_3\subseteq\cdots$$
      eventually stabilizes, i.e., there exists some $N$ with $I_N=I_{N+1}=I_{N+2}=\cdots$.
    \end{definition} 
    \begin{block}{Examples}
      \begin{itemize}
      \item Any field $k$
      \item $\Z$ or more generally any principle ideal domain
      \item $R$ Noetherian $\implies R/I$ Noetherian
        \end{itemize}
      \end{block}

  
    \end{frame}

  \begin{frame}{Why study Noetherian Rings?}
    Because of this lemma:
    \begin{lemma} A ring $R$ is Noetherian if and only if every ideal is finitely generated.\end{lemma}
    \begin{proof}
      \begin{description}
      \item[$\implies$] Assume $I$ not f.g., try to generate, get contradiction.
      \item[$\impliedby$] \begin{itemize}
      \item Take an ascending chain $I_n$
      \item $I=\cup I_n$ is an ideal, hence $I=(r_1,\dots, r_k)$
      \item If $\{r_i\}\in I_n$, then $I_n=I$,
        \end{itemize}
        \end{description}

      \end{proof}
   
    \end{frame}

  \begin{frame}{Hilbert Basis Theorem}
    \begin{theorem} If $R$ is Noetherian, then so is $R[x]$.
    \end{theorem}
    \begin{proof}Main ideas: look at leading coefficients, induct on degree.
      \end{proof}
    \begin{corollary} Let $k$ be a field or principle ideal domain.  Then every ideal $I\subset k[x_1,\dots, x_n]$ is finitely generated.
    \end{corollary}
    \begin{proof} $k$-Noetherian $\implies k[x_1,\dots, x_n]$ Noetherian \\ $\implies$ every ideal of $k[x_1,\dots, x_n]$ is finitely generated
        \end{proof}
    \begin{block}{Hence: finitely generated $k$-algebras are finitely presented.}
      \end{block}
\end{frame}  
\end{document}
