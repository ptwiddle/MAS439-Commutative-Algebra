\documentclass{beamer}

\usepackage{amsmath, amssymb, hyperref, graphics, wasysym}
\usepackage{mathpazo}

\newcommand{\C}{\mathbb{C}}
\newcommand{\Z}{\mathbb{Z}}
\newcommand{\R}{\mathbb{R}}

\title{MAS439 Lecture 8 \\ Maximal, Prime, Radical}

\date{October 129h}

\begin{document}

\begin{frame}
\titlepage
\end{frame}


\begin{frame}{}

Last session we defined quotient rings, and proved the universal property of quotient rings and the isomorphism theorems. \\~\\

Now we ask the following question: what conditions on $I$ make $R/I$ nice?  Specifically, when is $R/I$ a field/integral domain/reduced?  \\~\\

Tomorrow we will introduce the notion of \emph{algebras}.


\end{frame}

\begin{frame}[plain, c]

\huge
\begin{center}


{\usebeamercolor[fg]{frametitle} $R/I$ is a field $\iff$ $I$ is maximal}
$R/I$ is a domain $\iff$ $I$ is prime
$R/I$ is a reduced $\iff$ $I$ is radical

\end{center}
\end{frame}


\begin{frame}{An observation about fields}

\begin{lemma} A ring $R$ is a field if and only if the only ideals are $\{0\}$ and $R$.
\end{lemma}

\begin{proof}
Suppose $R$ a field, and $I\neq\{0\}$ an ideal.  We must show $I=R$.

\begin{itemize}
\item $\exists 0 \neq r\in I$. 

\item  Since $R$ a field, $\exists s\in R$ s.t. $rs=1$.

\item Since $I$ ideal, $r\in I$, we have $1=s\cdot r\in I$

\item But then $I=R$

\end{itemize}
\end{proof}
\end{frame}


\begin{frame}

\begin{lemma} A ring $R$ is a field if and only if the only ideals are $\{0\}$ and $R$.
\end{lemma}
\begin{proof}
Suppose the only ideals of $R$ are $\{0\}$ and $R$, and let $0\neq r\in R$.  We must show $r$ is a unit.

Consider $(r)$, the ideal generated by $r$. 
\begin{itemize}
\item  Since $r\in(r), (r)\neq\{0\}$.  
\item Hence $(r)=R$.  
\item Hence $1\in (r)$
\item So $1=r\cdot s$
\end{itemize}


\end{proof}

\end{frame}

\begin{frame}{Maps out of fields are injective\footnote{Terms and conditions may apply}}
\begin{lemma} Let $R$ be a field, and $\varphi:R\to S$ a homomorphism.  Then either
\begin{enumerate}
\item $\varphi$ is injective
\item $S$ is the trivial ring
\end{enumerate}
\end{lemma}

\begin{proof}
We have $\ker(\varphi)$ is either $\{0\}$, in which case $\varphi$ is injective, or $\ker(\varphi)=R$, in which case $1_S=0_S$.
\end{proof}

\end{frame}

\begin{frame}{Now we can prove what we wanted }
By the second isomorphism theorem, ideals in $R/I$ are of the form $J/I$, with $I\subset J\subset R$ an ideal.

\begin{definition}
A proper ideal $I$ is \emph{maximal} if there are no ideals $J$ with $I\subsetneq J\subsetneq R$
\end{definition}

\begin{lemma}
Let $R/I$ is a maximal ideal if and only if $R/I$ is a field.
\end{lemma}

Note: maximal ideals are often written $\mathfrak{m}$ ({\tt \textbackslash mathfrak\{m\}})


\end{frame}





\begin{frame}{Maximal ideals of $\mathbb{Z}$}

\begin{itemize}
\item Ideals of $\mathbb{Z}$ are of the form $(n)$.
\item $(n)\subset (m)$ if and only if $m$ divides $n$
\item So $(n)$ is maximal if and only if the element $n$ is prime
\end{itemize}

Indeed, we have seen $\mathbb{Z}/n\mathbb{Z}$ is a field if and only if $n$ is prime.

\end{frame}

\begin{frame}{Maximal ideals always exist}
\begin{lemma} Let $r\in R$ not be a unit.  Then $r$ is contained in some maximal ideal $\mathfrak{m}\subset I$
\end{lemma}

\begin{proof} Consider the set of all proper ideals of $R$ that contain $r$, ordered under inclusion.  Suppose 
$$r\in I_1\subset I_2\subset\cdots I_n\subset \cdots $$
is a chain of ideals.  Then we can see $\cup I_n$ is a proper ideal containing $r$.  We now apply Zorn's lemma.
\end{proof}

\end{frame}


\begin{frame}[plain, c]


\huge
\begin{center}

$R/I$ is a field $\iff$ $I$ is maximal
{\usebeamercolor[fg]{frametitle} $R/I$ is a domain $\iff$ $I$ is prime}

$R/I$ is a reduced $\iff$ $I$ is radical

\end{center}
\end{frame}



\begin{frame}{Once you define prime ideals, it's obvious}
\begin{definition}
An ideal $I\subset R$ is \emph{prime} if $ab\in I$ implies $a\in I$ or $b\in I$
\end{definition}

\begin{lemma}
An ideal $I$ is prime if and only if $R/I$ is an integral domain.
\end{lemma}

\begin{proof}
For $a\in R$, let $[a]$ denote the image of $a$ in $R/I$. \\~\\

Then $a\in I\iff [a]=0$.  And $ a\cdot b\in I\iff [a]\cdot [b]=0$.  \\~\\

So the definition of $I$ being prime is exactly equivalent to $R/I$ being an integral domain.

\end{proof}


\end{frame}


\begin{frame}{Two little comments}

\begin{enumerate}
\item 
Note that $R$ being an integral domain is equivalent to $\{0\}\subset R$ being a prime ideal.

\item Sometimes you'll see a prime ideal denote $\mathfrak{p}$, (i.e., {\tt \textbackslash mathfrak\{p\}}), but it's a bit old-school now, in contrast to $\mathfrak{m}$ for a maximal ideal, which is still commonplace

\end{enumerate}

\end{frame}

\begin{frame}{Example: Prime ideals in $\mathbb{Z}$}

Remember all ideals in $\mathbb{Z}$ are principal, hence of the form $(n)$. \\~\\

An element $m\in\mathbb{Z}\in (n)$ if and only if $m=an$.
 
That is, if and only if $n$ divides $m$. \\~\\

So asking $(n)$ to be prime is asking for $n|ab\implies n|a$ or $n|b$.

 That is, asking for $n$ to be prime.


\end{frame}


\begin{frame}{Wait!  In $\mathbb{Z}$ nearly all prime ideals are maximal?}

Note that since all fields are integral domains, we have that all maximal ideals are prime. \\~\\

In $\mathbb{Z}$, the converse is nearly true -- the maximal ideals are the ideals $(p)$, with $p$ prime; the prime ideals in $\mathbb{Z}$ are $(p)$ and $(0)$. \\~\\

Essentially the same proof holds true in any principal ideal domain...

\end{frame}


\begin{frame}[plain, c]


\huge
\begin{center}

$R/I$ is a field $\iff$ $I$ is maximal
$R/I$ is a domain $\iff$ $I$ is prime
{\usebeamercolor[fg]{frametitle} $R/I$ is a reduced $\iff$ $I$ is radical}

\end{center}
\end{frame}

\begin{frame}{Some pun about radical ideals}

Recall that if $I$ is an ideal, the \emph{radical} of $I$ was
$$\sqrt{I}=\{a : a^n\in I\text{ for some } a\}$$

\begin{definition}
We call an ideal \emph{radical} if $I=\sqrt{I}$.  That is, $I$ is radical if and only if $a^n\in I\implies a\in I$.
\end{definition}

\begin{lemma}
$I$ is radical if and only if $\sqrt{I}$ is reduced
\end{lemma}

Note: $R/I$ being reduced is equivalent to $\{0\}$ being radical.

\end{frame}


\begin{frame}{Radical ideals in $\mathbb{Z}$}

\begin{lemma} The ideal $(n)$ is radical if and only if $n$ is square-free -- that is, $n$ has no repeated prime factors.
\end{lemma}

\begin{proof}
We have that $a\in \sqrt{(n)}$ if and only if $n$ divides $a^k$ for some $k$, if and only if $a$ contains all the prime factors of $n$.  \\~\\

For $(n)$ to be radical, this needs to be equivalent to $a$ dividing $n$, i.e., every prime factor of $n$ occuring exactly once.

\end{proof}

\end{frame}



\end{document}
